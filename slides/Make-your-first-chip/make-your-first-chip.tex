\documentclass[20pt]{beamer}


\usetheme{hm}

\setbeamertemplate{section in toc}[sections numbered]
\setbeamertemplate{subsection in toc}[subsections numbered]

% Subsecciones (indentadas)
\setbeamertemplate{subsection in toc}{
  \hspace*{3.0em}% <<< INDENTACIÓN
  \llap{\inserttocsectionnumber.\inserttocsubsectionnumber\ }%
  \inserttocsubsection\par
}

\usepackage{minted}
\usepackage{tikz}
\usepackage{caption}
\usepackage{subfigure}
\usepackage{tcolorbox}

\usemintedstyle{friendly}

% Kein Dekor auf Titelfolie
%\usetheme[nodecotitle]{hm}

% Kein Dekor auf anderen Folien
%\usetheme[nodeco]{hm}

% Logo im Titel und nicht unten (für wide&nodeco interessant)
%\usetheme[logotop]{hm}

% Keine Foliennummer
%\usetheme[noframenum]{hm}

% Creative Commons CC-BY Lizenz
%\usetheme[ccby]{hm}

\author{Daniel Arevalos}
\institute{Fakultät 07 -- Hochschule München}
\title{Implementing the Mini-Calculator ASIC}
%\title{Vorlesung I mit einem langen Titel über zwei Zeilen}
%\subtitle{A practical guide to chip design flow}
%\subtitle{Kapitel 1 -- Eine Einführung die über zwei Zeilen geht und trotzdem gut passt}

\begin{document}


\begin{frame}[plain]
\titlepage
\end{frame}

\begin{frame}[fragile]
\frametitle{Hot Fix}
We need yet another instalation inside the librelane nix-shell.
\begin{minted}[fontsize=\small]{bash}
$ apt update 
$ apt install -y xdot graphviz
\end{minted}
\end{frame}

\begin{frame}[fragile]
\frametitle{Synthesis}
So far we have the RTL of the mini-calculator. But now we want to make it an ASIC. Lets just take a look into what the synthesis looks like.

\begin{enumerate}
\item Go into the firstdesign/M2\_BCD/src/ folder and then run:
\begin{minted}[fontsize=\small]{bash}
$ yosys
$ show
\end{minted}
What are we looking at ?
\item Now inside yosys run:
\begin{minted}[fontsize=\small]{bash}
$ synth
$ show
\end{minted}
\end{enumerate}

\end{frame}



\end{document}