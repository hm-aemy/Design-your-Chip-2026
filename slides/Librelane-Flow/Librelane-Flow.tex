\documentclass[20pt]{beamer}


\usetheme{hm}

\setbeamertemplate{section in toc}[sections numbered]
\setbeamertemplate{subsection in toc}[subsections numbered]

% Subsecciones (indentadas)
\setbeamertemplate{subsection in toc}{
  \hspace*{3.0em}% <<< INDENTACIÓN
  \llap{\inserttocsectionnumber.\inserttocsubsectionnumber\ }%
  \inserttocsubsection\par
}

\usepackage{minted}
\usepackage{tikz}
\usepackage{caption}
\usepackage{subfigure}
\usepackage{tcolorbox}

% Kein Dekor auf Titelfolie
%\usetheme[nodecotitle]{hm}

% Kein Dekor auf anderen Folien
%\usetheme[nodeco]{hm}

% Logo im Titel und nicht unten (für wide&nodeco interessant)
%\usetheme[logotop]{hm}

% Keine Foliennummer
%\usetheme[noframenum]{hm}

% Creative Commons CC-BY Lizenz
%\usetheme[ccby]{hm}

\author{Daniel Arevalos}
\institute{Fakultät 07 -- Hochschule München}
\title{Librelane Flow}
%\title{Vorlesung I mit einem langen Titel über zwei Zeilen}
\subtitle{A practical guide to chip design flow}
%\subtitle{Kapitel 1 -- Eine Einführung die über zwei Zeilen geht und trotzdem gut passt}

\begin{document}

\begin{frame}[plain]
\titlepage
\end{frame}

\setlength{\parskip}{0mm}
\begin{frame}
\frametitle{Table of Contents}
\vspace{3cm}
\tableofcontents
\vspace{2cm}
\end{frame}

\section{What is Librelane?}

\begin{frame}
\frametitle{What is Librelane?}
\vspace{5mm}
\raggedright
An open-source, configurable digital ASIC flow built on open-source EDA tools, designed for education and real-world chip design.
{
\centering
\includegraphics[width=0.85\linewidth]{img/configurable_flow.png}
}
\vspace{1mm}
{\footnotesize
\textbf{More Information on:}
\href{https://ocdcpro.github.io/educator-portal/handbook/digital_flow/librelane/}{Librelane}, \href{https://ocdcpro.github.io/educator-portal/handbook/tools/openroad/}{OpenROAD}
}
\end{frame}

\section{Architectural Overview}

\begin{frame}
  \frametitle{Architectural Overview}

  \vspace{10mm}
  \centering
  \includegraphics[width=0.75\linewidth]{img/librelane-overview.png}

\end{frame}

\section{Librelane Flow}

\subsection{Config File}

\begin{frame}[fragile]
\frametitle{Config File}

\begin{columns}
\begin{column}{0.3\textwidth}
\begin{minted}[fontsize=\footnotesize, highlightlines={2-5},
highlightcolor=red!15]{json}
{
  "DESIGN_NAME": "my_design",
  "VERILOG_FILES": [
    "src/my_design.v"
  ],
\end{minted}
\begin{minted}[fontsize=\footnotesize, highlightlines={1-2},
highlightcolor=blue!15]{json}
  "CLOCK_PORT": "clk",
  "CLOCK_PERIOD": 10,

\end{minted}
\begin{minted}[fontsize=\footnotesize, highlightlines={1-2},
highlightcolor=cyan!15]{json}
  "FP_CORE_UTIL": 50,
  "FP_ASPECT_RATIO": 1.0
\end{minted}
\begin{minted}[fontsize=\footnotesize, highlightlines={2},
highlightcolor=green!15]{json}

  "PL_TARGET_DENSITY": 0.55,
}
\end{minted}
\vspace{10mm}
    {\footnotesize
    \textbf{More Information on:}
    \href{https://ocdcpro.github.io/educator-portal/handbook/digital_flow/librelane/#config-file}{Config File}
    }

\end{column}
\begin{column}{0.6\textwidth}
\begin{itemize}
  \item \textcolor{red!60}{Design Definition:} Defines the logical identity of the design and specifies the RTL sources that will be used as input for the flow. 
  \item \textcolor{blue!60}{Clock Definition:} Describes the main clock interface and timing target that guide synthesis, timing analysis, and optimization throughout the flow.
  \item \textcolor{cyan!60}{Floorplanning (High-level):} Sets high-level constraints that shape the core area, controlling utilization and overall layout proportions.
  \item \textcolor{green!100}{Placement Control:} Guides how densely standard cells are packed during placement to balance area efficiency and routability.
\end{itemize}
\end{column}
\end{columns}
\end{frame}

\subsection{Macros}


\begin{frame}[fragile]
\frametitle{Macros}
\begin{columns}
\begin{column}{0.5\textwidth}
\begin{minted}[fontsize=\scriptsize]{json}
"MACROS": {
  "your_module": {
    "gds": "path/your_module.gds",
    "lef": "path/your_module.lef",
    "nl":  "path/your_module.pnl.v",

    "lib": {
      "nom_typ_1p20V_25C":  "path/your_module.lib",
      "nom_fast_1p32V_m40C": "path/your_module.lib",
      "nom_slow_1p08V_125C": "path/your_module.lib"
    },

    "spef": {
      "nom_typ_1p20V_25C": "path/your_module.spef"
    },

    "instances": {
      "inst_0": { "location": [640, 630], "orientation": "N" },
      "inst_1": { "location": [440, 430], "orientation": "N" }
    }
  }
}
\end{minted}
\end{column}
\begin{column}{0.5\textwidth}
\begin{itemize}
  \item A physical abstract view (LEF) to define area, pins, and routing blockages.
  \item A final layout (GDS) used for stream-out and sign-off verification.
  \item A logical netlist to connect the macro at the top level.
  \item Timing models across relevant corners for chip-level timing analysis.
  \item Optional parasitic information to improve accuracy during sign-off.
  \item Fixed placement information when the macro must be placed explicitly.
\end{itemize}
\end{column}
\end{columns}
\end{frame}

\subsection{Classic vs Chip flow}

\begin{frame}[fragile]
\frametitle{Classic vs Chip flow}

\begin{columns}
\begin{column}{0.5\textwidth}
\begin{minted}[fontsize=\footnotesize]{json}
{
"meta": {
        "version": 3,
        "flow": "Classic"
    },
...
}
\end{minted}
\centering
\includegraphics[width=0.7\linewidth]{img/counter_gds.png}
%\begin{itemize}
%  \item Generally used to harden a single design block from RTL to GDS.
%  \item Assumes a self-contained design.
%\end{itemize}
\end{column}
\begin{column}{0.5\textwidth}
\begin{minted}[fontsize=\footnotesize]{json}
{
"meta": {
        "version": 3,
        "flow": "Chip"
    },
...
}
\end{minted}
\centering
\includegraphics[width=0.7\linewidth]{img/counter_chip_gds.png}
%\begin{itemize}
%  \item Generally used to integrate multiple hardened macros into a full chip.
%  \item Focuses on chip-level constraints and padring.
%\end{itemize}
\end{column}
\end{columns}
\end{frame}

\subsection{Steps}

\begin{frame}
\frametitle{Steps}

\begin{columns}
\begin{column}{0.7\textwidth}
\begin{itemize}
  \item The flow is organized as a sequence of well-defined steps, each performing a specific transformation or verification task.
  \item Each step operates on the design state, progressively refining it from RTL to a manufacturable layout.
  \item Steps can include analysis, transformation, validation, or checking stages.
  \item The execution order is predefined, but individual steps can be enabled, disabled, or customized.  
  \item Intermediate results are stored, allowing inspection, debugging, and visualization at any stage of the flow.
\end{itemize}

\end{column}
\begin{column}{0.3\textwidth}

\centering
\includegraphics[width=0.4\linewidth]{img/librelane-steps.png}

\end{column}
\end{columns}

\end{frame}

\section{Quick tutorial}

\subsection{Classic Flow}

\begin{frame}[fragile]
\frametitle{Quick tutorial - Classic Flow}

\begin{columns}
\begin{column}{0.5\textwidth}

\vspace{10mm}
\textcolor{red!100}{counter\_8bit.v}
\begin{minted}[fontsize=\footnotesize]{verilog}
{
module counter_8bit (
	input  logic       clk_i,
	input  logic       rst_ni,
	output logic [7:0] count_o
);

	always_ff @(posedge clk_i) begin
        	if (!rst_ni) begin
            	count_o <= '0;
        	end else begin
            	count_o <= count_o + 1;
        	end
	end

endmodule
}
\end{minted}
\end{column}
\begin{column}{0.5\textwidth}
\textcolor{red!100}{config.json}
\begin{minted}[fontsize=\footnotesize]{json}
{
    "DESIGN_NAME": "counter_8bit",
    "VERILOG_FILES": [
        "src/counter_8bit.v"
    ],

    "CLOCK_PORT": "clk",
    "CLOCK_PERIOD": 10,

    "FP_CORE_UTIL": 50,
    "FP_ASPECT_RATIO": 1.0,
    
    "PL_TARGET_DENSITY": 0.55
}
\end{minted}
\vspace{10mm}
\textcolor{red!100}{CLI}
\begin{minted}[fontsize=\footnotesize]{bash}
export PDK=ihp-sg13g2
librelane -pdk ihp-sg13g2 config.json
\end{minted}

\end{column}
\end{columns}

\end{frame}

\begin{frame}[fragile]
\frametitle{Visualization of Results}
\begin{columns}
\begin{column}{0.5\textwidth}
\begin{minted}[fontsize=\footnotesize]{bash}
librelane --last-run config.json --flow OpenInOpenROAD
\end{minted}
\begin{minted}[fontsize=\tiny, breaklines]{text}
runs/<run_tag>
|-- final
|-- tmp
|-- error.log
|-- info.log
|-- resolved.json
|-- warning.log
|-- 01-verilator-lint
|-- 02-checker-linttimingconstructs
|-- 03-checker-linterrors
|-- 04-yosys-jsonheader
|-- 05-yosys-synthesis
|-- 06-checker-yosysunmappedcells
|-- 07-checker-yosyssynthchecks
|-- 08-openroad-checksdcfiles
|-- 09-openroad-staprepnr
|-- 10-openroad-floorplan
|-- 11-odb-setpowerconnections
|-- 12-odb-manualmacroplacement
|-- 13-openroad-cutrows
|-- 14-openroad-tapendcapinsertion
|-- 15-openroad-globalplacementskipio
.
.
.
\end{minted}

\end{column}
\begin{column}{0.5\textwidth}

\centering
\includegraphics[width=1\linewidth]{img/openroad_gui.png}

\end{column}
\end{columns}

\end{frame}

\subsection{Chip Flow}

\begin{frame}
  \frametitle{Quick tutorial - Chip Flow}
\begin{columns}
\begin{column}{0.5\textwidth}
\begin{itemize}
  \item Full-chip design integrating two 8-bit counter macros.
  \item Macros are reused from the Classic flow and instantiated at the top level.
  \item Shared clock and reset, producing a 16-bit output.
  \item Dedicated core and IO power pads.
\end{itemize}
\end{column}
\begin{column}{0.5\textwidth}

\centering
\includegraphics[width=1\linewidth]{img/counter_chip_openroad.png}

\end{column}
\end{columns}
\end{frame} 


%\subsection{Modifying Steps}

%\begin{frame}
%\frametitle{Modifying Steps}
%the idea here is that I show that you can modify most of the steps for example the OpenROAD scripts in tcl. example the pdn.tcl -> like was necessary for the custom power rings for the SRAM macros.

%Quizas sea una buena idea ponerle rings custom a uno de los contadores, para que asi luego muestre como tuve que modificar el TCL. 

%Crear una narrativa con el tema de modificar los scripts y el segundo ejemplo mas complejo con el flow Chip, y no classic.
%\end{frame}

\end{document}