\documentclass[20pt]{beamer}


\usetheme{hm}

\setbeamertemplate{section in toc}[sections numbered]
\setbeamertemplate{subsection in toc}[subsections numbered]

% Subsecciones (indentadas)
\setbeamertemplate{subsection in toc}{
  \hspace*{3.0em}% <<< INDENTACIÓN
  \llap{\inserttocsectionnumber.\inserttocsubsectionnumber\ }%
  \inserttocsubsection\par
}

\usepackage{minted}
\usepackage{tikz}
\usepackage{caption}
\usepackage{subfigure}

% Kein Dekor auf Titelfolie
%\usetheme[nodecotitle]{hm}

% Kein Dekor auf anderen Folien
%\usetheme[nodeco]{hm}

% Logo im Titel und nicht unten (für wide&nodeco interessant)
%\usetheme[logotop]{hm}

% Keine Foliennummer
%\usetheme[noframenum]{hm}

% Creative Commons CC-BY Lizenz
%\usetheme[ccby]{hm}

\author{Daniel Arevalos}
\institute{Fakultät 07 -- Hochschule München}
\title{Chip Design \& Implementation}
%\title{Vorlesung I mit einem langen Titel über zwei Zeilen}
\subtitle{From RTL to GDSII}
%\subtitle{Kapitel 1 -- Eine Einführung die über zwei Zeilen geht und trotzdem gut passt}


\begin{document}

\begin{frame}[plain]
\titlepage
\end{frame}

\setlength{\parskip}{0mm}
\begin{frame}
\frametitle{Table of Contents}
\vspace{3cm}
\tableofcontents
\vspace{2cm}

\begin{tikzpicture}[overlay, remember picture]

% Llave FRONT-END (secciones 1–3)
\draw[
  decorate,
  decoration={brace, amplitude=8pt},
  line width=1.2pt
]
  (9cm,13.7cm) -- (9cm,11.9cm)
  node[midway, right=10pt]{\large Front-End};

% Llave BACK-END (secciones 4–5)
\draw[
  decorate,
  decoration={brace, amplitude=8pt},
  line width=1.2pt
]
  (9cm,11.5cm) -- (9cm,4cm)
  node[midway, right=10pt]{\large Back-End};

\end{tikzpicture}

\end{frame}

\section{Introduction}

\begin{frame}
\frametitle{Introduction}
\begin{columns}
    \begin{column}{0.38\textwidth}
        \begin{itemize}
            \item Bullet 1
            \item Bullet 2
            \item Bullet 3
        \end{itemize}
    \end{column}
    \begin{column}{0.6\textwidth}
      \centering
      \includegraphics[width=0.39\textwidth]{img/VLSI-flow-v2.png}
    \end{column}
\end{columns}
\end{frame}

\begin{frame}[plain]
\vfill
\centering
{\LARGE Front-End}
\vfill
\end{frame}

\section{RTL Design}

\begin{frame}
\frametitle{RTL Design}
\end{frame}

\section{Simulation \& Verification}

\begin{frame}
  \frametitle{Simulation \& Verification}
\end{frame}

\section{Physical Implementation}

\begin{frame}[plain]
\vfill
\centering
{\LARGE Back-End}
\vfill
\end{frame}

\subsection{Synthesis}
\begin{frame}
\frametitle{Synthesis - Standard Cells}
\begin{columns}
    \begin{column}{0.3\textwidth}
        \begin{itemize}
            \item Bullet 1
            \item Bullet 2
            \item Bullet 3
        \end{itemize}
    \end{column}

    \begin{column}{0.68\textwidth}
        \centering
        \includegraphics[width=0.3\textwidth]{img/and2.png}
    \end{column}
  \end{columns}

\end{frame}

\begin{frame}
  \frametitle{Synthesis - Inputs \& Outputs}
  \begin{columns}
    \begin{column}{0.3\textwidth}
        \begin{itemize}
            \item Bullet 1
            \item Bullet 2
            \item Bullet 3
        \end{itemize}
    \end{column}

    \begin{column}{0.68\textwidth}
        \centering
        \includegraphics[width=1\textwidth]{img/synthesis.png}
    \end{column}
  \end{columns}
\end{frame}

\begin{frame}[fragile]
\frametitle{Synthesis - How it looks in the Practice?}
\begin{columns}
\begin{column}{0.33\textwidth}
\begin{minted}[fontsize=\scriptsize,frame=single]{verilog}
module example (
    input  wire clk,
    input  wire rst,
    output wire y
);
    assign y = clk & ~rst;
endmodule
\end{minted}
\captionof{listing}{RTL code}
\end{column}
\begin{column}{0.33\textwidth}
\begin{minted}[fontsize=\scriptsize,frame=single]{verilog}
module fulladd(a, b, c_in, c_out, sum);
  wire _00_;
  wire _01_;
  wire _02_;
  ...
  input [3:0] a;
  wire [3:0] a;
  ...
  assign _14_ = _07_ & ~(_13_);
  assign _15_ = _06_ & ~(_14_);
  assign c_out = _15_ | _04_;
  assign sum[0] = ~(_10_ ^ c_in);
  assign sum[1] = ~(_12_ ^ _08_);
  assign sum[2] = _14_ ^ _05_;
  assign _16_ = ~(_14_ | _05_);
  assign _17_ = _16_ | ~(_02_);
  assign sum[3] = _17_ ^ _01_;
endmodule
\end{minted}
\captionof{listing}{Netlist after synthesis}
\end{column}
\begin{column}{0.34\textwidth}
\begin{minted}[fontsize=\scriptsize,frame=single]{verilog}
module fulladd(a, b, c_in, c_out, sum);
  wire _00_;
  wire _01_;
  wire _02_;
  ...
  output [3:0] sum;
  wire [3:0] sum;
  sg13g2_xor2_1 _61_ (
    .A(_25_),.B(_21_),.X(_42_)
  );
  sg13g2_o21ai_1 _62_ (
    .A1(_40_),.A2(_41_),.B1(_28_),.Y(_27_)
  );
  sg13g2_xnor2_1 _63_ (
    .A(_26_),.B(_35_),.Y(_43_)
  );
  sg13g2_xnor2_1 _64_ (
    .A(_36_),.B(_38_),.Y(_44_)
  );
  sg13g2_xnor2_1 _65_ (
    .A(_31_),.B(_39_),.Y(_45_)
  );
  ...
endmodule

\end{minted}
\captionof{listing}{Netlist after synthesis}
\end{column}
\end{columns}
\begin{tikzpicture}[overlay, remember picture]
\draw[->, line width=4pt] (6.8cm,7.7cm) -- (8cm,7.7cm);
\draw[->, line width=4pt] (17.5cm,7.7cm) -- (18.8cm,7.7cm);
\end{tikzpicture}
\end{frame}


\subsection{Floorplanning}

\begin{frame}
\frametitle{Floorplanning}

\begin{columns}
    \begin{column}{0.3\textwidth}
        \begin{itemize}
            \item Bullet 1
            \item Bullet 2
            \item Bullet 3
        \end{itemize}
    \end{column}

    \begin{column}{0.68\textwidth}
        \centering
        \includegraphics[width=0.6\linewidth]{img/floorplan_init.png}
    \end{column}
  \end{columns}
\end{frame}

\begin{frame}
\frametitle{Floorplanning - Padring}

\begin{columns}
    \begin{column}{0.38\textwidth}
        \begin{itemize}
            \item Bullet 1
            \item Bullet 2
            \item Bullet 3
        \end{itemize}
    \end{column}
    \begin{column}{0.6\textwidth}
      \centering
      \includegraphics[width=0.9\linewidth]{img/padring.png}\par
      \includegraphics[width=0.9\linewidth]{img/esd_chain.png}
    \end{column}
\end{columns}
\end{frame}

\begin{frame}
\frametitle{Floorplanning - Power Grid}
\begin{columns}
    \begin{column}{0.3\textwidth}
        \begin{itemize}
            \item Bullet 1
            \item Bullet 2
            \item Bullet 3
        \end{itemize}
    \end{column}

    \begin{column}{0.68\textwidth}
        \centering
        \includegraphics[width=0.8\linewidth]{img/power_grid.png}
    \end{column}
  \end{columns}
\end{frame}

\subsection{Placement}

\begin{frame}
\frametitle{Placement}
\begin{columns}
  \begin{column}{0.45\textwidth}
    Global Placement
    \begin{itemize}
      \item qqasd
    \end{itemize}
    \centering
    \includegraphics[width=0.6\linewidth]{img/global_placement.png}
  \end{column}
  \begin{column}{0.45\textwidth}
    Detailed Placement
    \begin{itemize}
      \item asdasd
    \end{itemize}
    \centering
    \includegraphics[width=0.6\linewidth]{img/detailed_placement.png}
  \end{column}
\end{columns}
\end{frame}

\subsection{CTS}

\begin{frame}
\frametitle{Clock Tree Synthesis (CTS)}
\begin{columns}
  \begin{column}{0.45\textwidth}
    Before CTS
    \begin{itemize}
      \item qqasd
    \end{itemize}
    \centering
    \includegraphics[width=0.6\linewidth]{img/chip-before-cts.png}
  \end{column}
  \begin{column}{0.45\textwidth}
    After CTS
    \begin{itemize}
      \item asdasd
    \end{itemize}
    \centering
    \includegraphics[width=0.6\linewidth]{img/chip-after-cts.png}
  \end{column}
\end{columns}
\end{frame}

\subsection{Routing}

\begin{frame}
\frametitle{Routing}
\centering
    \includegraphics[width=0.6\linewidth]{img/routing.png}
\end{frame}

\section{Signoff \& Physical Verification}

\begin{frame}
\frametitle{Signoff \& Physical Verification}
\begin{enumerate}
  \item Timing Signoff
  \begin{enumerate}
    \item Fill insertion
    \item RC Extraction
    \item STA post Physical Implementation
  \end{enumerate}
  \item Physical Verification
  \begin{enumerate}
    \item DRC
    \item LVS
  \end{enumerate}
  \item Chip Finishing
  \begin{enumerate}
    \item Sealring
    \item Metal Fill
    \item Others...
  \end{enumerate}
\end{enumerate}
\end{frame}

\subsection{Timing Signoff}

\begin{frame}
\frametitle{Timing Signoff}
\begin{itemize}
  \item Fill Insertion
  \begin{itemize}
    \item 
  \end{itemize}
  \item RC Extraction
  \item STA post Physical Implementation
\end{itemize}
\end{frame}

\subsection{Layout (Physical) Verification}

\begin{frame}
\frametitle{Physical Verification - DRC}

\begin{columns}

\begin{column}{0.5\textwidth}
\begin{itemize}
  \item DRC run at the fullchip level to ensure that the layout adheres to the foundry's physical rules (geometries, spacing, etc.)
\end{itemize}
\end{column}

\begin{column}{0.45\textwidth}
\centering
\includegraphics[width=0.8\linewidth]{img/drc-sg13g2-1.png}
\includegraphics[width=0.8\linewidth]{img/drc-sg13g2.png}
\end{column}

\end{columns}

\end{frame}

\begin{frame}
  \frametitle{Physical Verification - LVS}

  \begin{itemize}
    \item Extract layout (GDS) and build a SPICE netlist
    \begin{itemize}
      \item Sometimes need to black-box sensitive layouts.
    \end{itemize}
    \item Export RTL and synthesized netlist from earlier stages.
    \item Compare both netlists to ensure they match.
  \end{itemize}

\end{frame}

\subsection{Chip Finishing}

\begin{frame}
  \frametitle{Chip Finishing}
\begin{itemize}
  \item Sealring
  \item Metal Fill
\end{itemize}
\end{frame}

\section{Tapeout \& Fabrication}

\begin{frame}
  \frametitle{Tapeout \& Fabrication}
\end{frame}

\end{document}
