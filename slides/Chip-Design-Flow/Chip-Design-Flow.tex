\documentclass[20pt]{beamer}


\usetheme{hm}

\setbeamertemplate{section in toc}[sections numbered]
\setbeamertemplate{subsection in toc}[subsections numbered]

% Subsecciones (indentadas)
\setbeamertemplate{subsection in toc}{
  \hspace*{3.0em}% <<< INDENTACIÓN
  \llap{\inserttocsectionnumber.\inserttocsubsectionnumber\ }%
  \inserttocsubsection\par
}

\usepackage{minted}
\usepackage{tikz}
\usepackage{caption}
\usepackage{subfigure}
\usepackage{tcolorbox}

% Kein Dekor auf Titelfolie
%\usetheme[nodecotitle]{hm}

% Kein Dekor auf anderen Folien
%\usetheme[nodeco]{hm}

% Logo im Titel und nicht unten (für wide&nodeco interessant)
%\usetheme[logotop]{hm}

% Keine Foliennummer
%\usetheme[noframenum]{hm}

% Creative Commons CC-BY Lizenz
%\usetheme[ccby]{hm}

\author{Daniel Arevalos}
\institute{Fakultät 07 -- Hochschule München}
\title{Chip Design \& Implementation}
%\title{Vorlesung I mit einem langen Titel über zwei Zeilen}
\subtitle{From RTL to GDSII}
%\subtitle{Kapitel 1 -- Eine Einführung die über zwei Zeilen geht und trotzdem gut passt}


\begin{document}

\begin{frame}[plain]
\titlepage
\end{frame}

\setlength{\parskip}{0mm}
\begin{frame}
\frametitle{Table of Contents}
\vspace{3cm}
\tableofcontents
\vspace{2cm}

\begin{tikzpicture}[overlay, remember picture]

% Llave FRONT-END (secciones 1–3)
\draw[
  decorate,
  decoration={brace, amplitude=8pt},
  line width=1.2pt
]
  (11.5cm,14cm) -- (11.5cm,12.3cm)
  node[midway, right=10pt]{\large Front-End};

% Llave BACK-END (secciones 4–5)
\draw[
  decorate,
  decoration={brace, amplitude=8pt},
  line width=1.2pt
]
  (11.5cm,12cm) -- (11.5cm,3.5cm)
  node[midway, right=10pt]{\large Back-End};

\end{tikzpicture}

\end{frame}

\section{Introduction}

\begin{frame}
\frametitle{Introduction}
\begin{columns}
    \begin{column}{0.6\textwidth}
        \begin{itemize}
            \item Digital IC design follows a structured flow from high-level specifications to fabrication.
            \item The flow is divided into Front-End (technology-independent) and Back-End (technology-dependent) stages.
            \item Front-End focuses on functionality and correctness (RTL design and verification).
            \item Back-End transforms the RTL into a physical layout that meets timing, power, and manufacturing constraints.
            \item The process is iterative, with verification and checks at multiple stages before tape-out.
        \end{itemize}
    \end{column}
    \begin{column}{0.38\textwidth}
      \centering
      \includegraphics[width=0.63\textwidth]{img/VLSI-flow-v2.png}
    \end{column}
\end{columns}
\end{frame}

\begin{frame}[plain]
\vfill
\centering
{\LARGE Front-End}
\vfill
\end{frame}

\section{RTL Design}

\begin{frame}[fragile]
\frametitle{RTL Design}

\begin{columns}
\begin{column}{0.5\textwidth}
\begin{itemize}
  \item Describes the hardware behavior using a register-transfer level (RTL) abstraction.
  \item Technology-independent and focused on functionality, not physical implementation.
  \item Serves as the main input for simulation, verification, and synthesis.
  \item \textcolor{red!60}{Module}, \textcolor{blue!60}{Body}, \textcolor{green!70}{Signal Declaration}, \textcolor{cyan!70}{Instantiation}, \textcolor{yellow!100}{Combinational Logic}
\end{itemize}
\vspace{15mm}
    {\footnotesize
    \textbf{More Information on:}
    \href{https://vlsi.ethz.ch/w/images/d/de/02_RTL.pdf}{Refresher on SystemVerilog}
    }

\end{column}
\begin{column}{0.65\textwidth}
\begin{tcolorbox}[
  colback=red!15,
  colframe=red!15,
  boxrule=0pt,
  arc=0pt,
  left=2mm,right=2mm,top=1mm,bottom=1mm
]
\begin{minted}[fontsize=\footnotesize]{verilog}
module top #(
parameter int Width = 16
) (
input logic clk_i,
input logic rst_ni,
input logic mode_i,
input logic [Width-1:0] data_in_i,
output logic [Width-1:0] result_o
);
\end{minted}
\end{tcolorbox}
\begin{tcolorbox}[
  colback=blue!20,
  colframe=blue!20,
  boxrule=0pt,
  arc=0pt,
  left=2mm,right=2mm,top=1mm,bottom=1mm
]
\begin{minted}[fontsize=\footnotesize, highlightlines={1-3},
highlightcolor=green!15]{verilog}
// Declare signals to be used in the module
logic [Width-1:0] first, second;
logic [Width-1:0] combine_and, combine_or;
\end{minted}
\begin{minted}[fontsize=\footnotesize, highlightlines={1-5},
highlightcolor=cyan!15]{verilog}
// instantiate two blocks with different names i_reg_1 and frank
ffs #(.Width(Width)) i_reg_1 (
.clk_i(clk_i), .rst_ni(rst_ni), .in_i(data_in_i), .out_o(first));
ffs #(.Width(Width)) frank (
.clk_i(clk_i), .rst_ni(rst_ni), .in_i(first), .out_o(second));
\end{minted}
\begin{minted}[fontsize=\footnotesize, highlightlines={1-5},
highlightcolor=yellow!15]{verilog}
// combine outputs
assign combine_and = first & second ;
assign combine_or = first | second ;
// assign result
assign result_o = mode_i ? combine_and : combine_or ;
\end{minted}
\end{tcolorbox}
\begin{tcolorbox}[
  colback=red!15,
  colframe=red!15,
  boxrule=0pt,
  arc=0pt,
  left=2mm,right=2mm,top=1mm,bottom=1mm
]
\begin{minted}[fontsize=\footnotesize]{verilog}
endmodule // top
\end{minted}
\end{tcolorbox}
\end{column}
\end{columns}
\end{frame}

\section{Simulation \& Formal Verification}

\begin{frame}
\frametitle{Simulation \& Formal Verification}
\begin{columns}
    \begin{column}{0.6\textwidth}
        \begin{itemize}
            \item Validates RTL functionality against the design specifications.
            \item Simulation checks behavior using testbenches and input stimuli.
            \begin{itemize}
              \item Applies stimuli to the Design Under Test (DUT) using testbenches.  
              \item Compares DUT outputs against a golden reference model.
              \item Detects functional mismatches and reports errors.
            \end{itemize}
            \item Formal verification proves correctness using mathematical models, without test vectors.
        \end{itemize}
    \vspace{15mm}
    {\footnotesize
    \textbf{More Information on:}
    \href{https://ocdcpro.github.io/educator-portal/handbook/concepts/simulation.html}{Simulation}, \href{}{Formal Verification}
    }
    \end{column}

    \begin{column}{0.5\textwidth}
        \centering
        \includegraphics[width=1\textwidth]{img/simualtion-process.png}
        \includegraphics[width=1\textwidth]{img/surfer.png}
    \end{column}
  \end{columns}
\end{frame}

\section{Physical Implementation}

\begin{frame}[plain]
\vfill
\centering
{\LARGE Back-End}
\vfill
\end{frame}

\subsection{Synthesis}
\begin{frame}
\frametitle{Synthesis - Standard Cells}
\begin{columns}
    \begin{column}{0.5\textwidth}
        \begin{itemize}
            \item Pre-designed and pre-characterized logic building blocks.
            \item Provided by the foundry or PDK as a standard cell library.
            \item Characterized for timing, power, and area.
            \item Used by synthesis tools to map RTL into hardware.
        \end{itemize}
    \end{column}

    \begin{column}{0.5\textwidth}
        \centering
        \includegraphics[width=0.8\textwidth]{img/and2.png}
    \end{column}
  \end{columns}

\end{frame}

\begin{frame}
  \frametitle{Synthesis - Inputs \& Outputs}
  \begin{columns}
    \begin{column}{0.4\textwidth}
        \begin{itemize}
            \item Inputs: RTL design (HDL) and design constraints.
            \item Uses standard cell and IP libraries for mapping.
            \item Translates RTL into a gate-level netlist.
            \item Preserves functional behavior while meeting timing and area constraints.
        \end{itemize}

\vspace{15mm}
    {\footnotesize
    \textbf{More Information on:}
    \href{https://ocdcpro.github.io/educator-portal/handbook/concepts/synthesis.html}{Synthesis}
    }

    \end{column}

    \begin{column}{0.6\textwidth}
        \centering
        \includegraphics[width=1.05\textwidth]{img/synthesis.png}
    \end{column}
  \end{columns}
\end{frame}

\begin{frame}[fragile]
\frametitle{Synthesis - How it looks in Practice?}
\begin{columns}
\begin{column}{0.33\textwidth}
\begin{minted}[fontsize=\scriptsize,frame=single]{verilog}
module example (
    input  wire clk,
    input  wire rst,
    output wire y
);
    assign y = clk & ~rst;
endmodule
\end{minted}
\captionof{listing}{RTL code}
\end{column}
\begin{column}{0.33\textwidth}
\begin{minted}[fontsize=\scriptsize,frame=single]{verilog}
module fulladd(a, b, c_in, c_out, sum);
  wire _00_;
  wire _01_;
  wire _02_;
  ...
  input [3:0] a;
  wire [3:0] a;
  ...
  assign _14_ = _07_ & ~(_13_);
  assign _15_ = _06_ & ~(_14_);
  assign c_out = _15_ | _04_;
  assign sum[0] = ~(_10_ ^ c_in);
  assign sum[1] = ~(_12_ ^ _08_);
  assign sum[2] = _14_ ^ _05_;
  assign _16_ = ~(_14_ | _05_);
  assign _17_ = _16_ | ~(_02_);
  assign sum[3] = _17_ ^ _01_;
endmodule
\end{minted}
\captionof{listing}{Netlist after initial synthesis}
\end{column}
\begin{column}{0.34\textwidth}
\begin{minted}[fontsize=\scriptsize,frame=single]{verilog}
module fulladd(a, b, c_in, c_out, sum);
  wire _00_;
  wire _01_;
  wire _02_;
  ...
  output [3:0] sum;
  wire [3:0] sum;
  sg13g2_xor2_1 _61_ (
    .A(_25_),.B(_21_),.X(_42_)
  );
  sg13g2_o21ai_1 _62_ (
    .A1(_40_),.A2(_41_),.B1(_28_),.Y(_27_)
  );
  sg13g2_xnor2_1 _63_ (
    .A(_26_),.B(_35_),.Y(_43_)
  );
  sg13g2_xnor2_1 _64_ (
    .A(_36_),.B(_38_),.Y(_44_)
  );
  sg13g2_xnor2_1 _65_ (
    .A(_31_),.B(_39_),.Y(_45_)
  );
  ...
endmodule
\end{minted}
\captionof{listing}{Netlist after technology mapping}
\end{column}
\end{columns}
\begin{tikzpicture}[overlay, remember picture]
\draw[->, line width=4pt] (6.8cm,7.7cm) -- (8cm,7.7cm);
\draw[->, line width=4pt] (17.5cm,7.7cm) -- (18.8cm,7.7cm);
\end{tikzpicture}
\end{frame}


\subsection{Floorplanning}

\begin{frame}
\frametitle{Floorplanning}

\begin{columns}
    \begin{column}{0.45\textwidth}
        \begin{itemize}
            \item The unit cost of IC is directly proportional to the silicon area.
            \item The netlist gives us the net cell area needed, but:
            \begin{itemize}
              \item Additional area is required to bring power and timing/clock signals to all cells.
            \end{itemize}
            \item I/O and packaging constraints must be considered.
            \item Macros (RAM, PLL, etc.) must be placed.
        \end{itemize}

\vspace{15mm}
    {\footnotesize
    \textbf{More Information on:}
    \href{https://ocdcpro.github.io/educator-portal/handbook/concepts/floorplanning/}{Floorplanning}
    }

    \end{column}

    \begin{column}{0.55\textwidth}
      \vspace{-7mm}
      \raggedright
      \includegraphics[width=1.1\linewidth]{img/floorplan_init.png}
    \end{column}
  \end{columns}
\end{frame}

\begin{frame}
\frametitle{Floorplanning - Padring}

\begin{columns}
    \begin{column}{0.38\textwidth}
        \begin{itemize}
            \item Physical connection to package.
            \item Electro Static Discharge (ESD) protection.
            \begin{itemize}
              \item Make sure that electrical discharge does not damage internal circuitry.
            \end{itemize}
            \item I/O Drivers.
            \item Types of I/O Cells.
            \begin{itemize}
              \item Digital I/O Buffers.
              \item Analog I/O Cells.
              \item Power/Ground Supply Pads.
            \end{itemize}
        \end{itemize}
    \end{column}
    \begin{column}{0.6\textwidth}
      \centering
      \includegraphics[width=1.05\linewidth]{img/padring.png}\par
      \includegraphics[width=1.1\linewidth]{img/esd_chain.png}
    \end{column}
\end{columns}
\end{frame}

\begin{frame}
\frametitle{Floorplanning - Power Grid}
\begin{columns}
    \begin{column}{0.4\textwidth}
        \begin{itemize}
            \item Power Rings. Distribute power from pads to core area.
            \item Power Stripes. Distribute power within core area.
            \item Power PADs. Connect power grid to standard cells.
        \end{itemize}
    \end{column}

    \begin{column}{0.6\textwidth}
        \centering
        \includegraphics[width=1\linewidth]{img/power_grid.png}
    \end{column}
  \end{columns}
\end{frame}

\subsection{Placement}

\begin{frame}
\frametitle{Global Placement}
\begin{columns}
  \begin{column}{0.45\textwidth}
    \begin{itemize}
      \item Distributes standard cells across the core using approximate positions.
      \item Optimizes wirelength and congestion at a global level.
      \item Produces a continuous placement (cells may overlap or be non-legal).
    \end{itemize}

\vspace{30mm}
    {\footnotesize
    \textbf{More Information on:}
    \href{https://ocdcpro.github.io/educator-portal/handbook/concepts/placement.html}{Placement}
    }

  \end{column}
  \begin{column}{0.45\textwidth}
    \centering
    \includegraphics[width=1\linewidth]{img/global_placement.png}
  \end{column}
\end{columns}
\end{frame}

\begin{frame}
\frametitle{Detailed Placement}
\begin{columns}
  \begin{column}{0.45\textwidth}
    \begin{itemize}
      \item Also known as legalization.
      \item Takes the global placement and fixes exact cell locations on rows.
      \item Performs local optimizations (shifts, swaps) to improve wirelength and timing while respecting rules.
    \end{itemize}

\vspace{30mm}
    {\footnotesize
    \textbf{More Information on:}
    \href{https://ocdcpro.github.io/educator-portal/handbook/concepts/placement.html}{Placement}
    }

  \end{column}
  \begin{column}{0.45\textwidth}
    \centering
    \includegraphics[width=1\linewidth]{img/detailed_placement.png}
  \end{column}
\end{columns}
\end{frame}

\subsection{CTS}

\begin{frame}
\frametitle{Clock Tree Synthesis (CTS)}
\begin{columns}
  \begin{column}{0.5\textwidth}
    \begin{itemize}
      \item Clock Tree Synthesis (CTS) builds the clock network.
      \item Ensures clock reaches all sequential elements.
      \item Minimizes clock skew and insertion delay.
      \item Guarantees correct timing across the design.
    \end{itemize}
  \end{column}
  \begin{column}{0.6\textwidth}
    \centering
    \includegraphics[width=0.47\linewidth]{img/chip-before-cts.png}
    \centering
    \includegraphics[width=0.52\linewidth]{img/cluster-before-cts.png}
    \centering
    \includegraphics[width=0.47\linewidth]{img/chip-after-cts.png}
    \centering
    \includegraphics[width=0.52\linewidth]{img/cluster-after-cts.png}
  \end{column}
\end{columns}
\end{frame}

\subsection{Routing}

\begin{frame}
\frametitle{Global Routing}
\begin{columns}
  \begin{column}{0.5\textwidth}
    \begin{itemize}
      \item Plans approximate net routes using a grid of routing cells (GCells).
      \item Models routing capacity and congestion across the chip.
      \item Produces routing guides, not final metal wires.
      \item Guides detailed routing toward a feasible solution.
\end{itemize}

\vspace{30mm}
    {\footnotesize
    \textbf{More Information on:}
    \href{https://ocdcpro.github.io/educator-portal/handbook/concepts/routing.html}{Routing}
    }


  \end{column}
  \begin{column}{0.45\textwidth}
    \vspace{-9mm}
    \raggedright
    \includegraphics[width=1.03\linewidth]{img/global_routing.png}
  \end{column}
\end{columns}
\end{frame}

\begin{frame}
\frametitle{Detailed Routing}
\begin{columns}
  \begin{column}{0.5\textwidth}
    \begin{itemize}
      \item Converts global routing guides into exact physical wires and vias.
      \item Assigns routes to specific tracks, layers, and vias.
      \item Strictly enforces all design rules (DRC): width, spacing, enclosures, min-area, etc.
      \item Produces a clean, manufacturable routing.
\end{itemize}

\vspace{20mm}
    {\footnotesize
    \textbf{More Information on:}
    \href{https://ocdcpro.github.io/educator-portal/handbook/concepts/routing.html}{Routing}
    }

  \end{column}
  \begin{column}{0.45\textwidth}
    \raggedright
    \includegraphics[width=1\linewidth]{img/detailed_routing_1.png}
  \end{column}
\end{columns}
\end{frame}

\section{Signoff \& Physical Verification}

\begin{frame}
\frametitle{Signoff \& Physical Verification}
\begin{enumerate}
  \item Timing Signoff
  \begin{enumerate}
    \item Fill insertion
    \item RC Extraction
    \item STA post Physical Implementation
  \end{enumerate}
  \item Physical Verification
  \begin{enumerate}
    \item DRC
    \item LVS
  \end{enumerate}
  \item Chip Finishing
  \begin{enumerate}
    \item Sealring
    \item Metal Fill
    \item Others...
  \end{enumerate}
\end{enumerate}
\end{frame}

\subsection{Timing Signoff}

\begin{frame}
\frametitle{Timing Signoff}

\begin{columns}
\begin{column}{0.33\textwidth}
  Fill Insertion
\begin{itemize}
  \item Inserted to maintain well and power rail continuity.
  \item Do not implement logic; purely physical cells.
\end{itemize}
\centering
\includegraphics[width=1\linewidth]{img/fillers.png}

\raggedright
\vspace{5mm}
    {\footnotesize
    \textbf{More Information on:}
    \href{https://ocdcpro.github.io/educator-portal/handbook/concepts/parasitics_extraction/}{RC Extraction}
    }

\end{column}
\begin{column}{0.33\textwidth}
RC Extraction
\begin{itemize}
  \item Extracts resistance and capacitance (RC) from the routed layout.
  \item Models parasitic effects of wires and vias.
\end{itemize}
\centering
\includegraphics[width=0.6\linewidth]{img/rc-layout.png}
\end{column}
\begin{column}{0.33\textwidth}
STA post Physical Implementation
\begin{itemize}
  \item Analyzes timing using post-layout parasitics (RC).
\end{itemize}
\centering
\includegraphics[width=1\linewidth]{img/RC-timing_1.png}
\includegraphics[width=1\linewidth]{img/RC-timing_2.png}
\end{column}
\end{columns}

\end{frame}

\subsection{Layout (Physical) Verification}

\begin{frame}
\frametitle{Physical Verification - DRC}

\begin{columns}
\begin{column}{0.5\textwidth}
\begin{itemize}
  \item DRC run at the fullchip level to ensure that the layout adheres to the foundry's physical rules (geometries, spacing, etc.)
\end{itemize}
\centering
\includegraphics[width=0.9\linewidth]{img/drc-sg13g2-1.png}
\end{column}

\begin{column}{0.5\textwidth}
\centering
\includegraphics[width=1\linewidth]{img/drc-sg13g2.png}
\end{column}
\end{columns}

\end{frame}

\begin{frame}
\frametitle{Physical Verification - LVS}

\begin{columns}
\begin{column}{0.5\textwidth}
\begin{itemize}
    \item Extract layout (GDS) and build a SPICE netlist
    \begin{itemize}
      \item Sometimes need to black-box sensitive layouts.
    \end{itemize}
    \item Export RTL and synthesized netlist from earlier stages.
    \item Compare both netlists to ensure they match.
  \end{itemize}
\end{column}

\begin{column}{0.55\textwidth}
\centering
\includegraphics[width=1.1\linewidth]{img/LVS.png}
\end{column}
\end{columns}
\end{frame}

\subsection{Chip Finishing}

\begin{frame}
\frametitle{Chip Finishing}

\begin{columns}
\begin{column}{0.6\textwidth}
Sealring
\begin{itemize}
  \item Protective structure placed around the chip perimeter.
  \item Helps isolate the core circuitry during dicing and packaging.
\end{itemize}
\centering
\includegraphics[width=0.5\linewidth]{img/sealring.png}
\end{column}

\begin{column}{0.45\textwidth}
Metal Fill
\begin{itemize}
  \item The wafer surface is polished to ensure flatness.
  \item Metal fill is added to areas of low metal density to even out the overall density across the chip.
\end{itemize}
\centering
\includegraphics[width=1\linewidth]{img/dishing-erosion.png}
\end{column}
\end{columns}

\end{frame}

\section{Tapeout \& Fabrication}

\begin{frame}
  \frametitle{Tapeout \& Fabrication}
\end{frame}

\end{document}
